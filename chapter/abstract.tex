\documentclass[/Users/ikedahajime/GitHub/reserch/master_report/thesis]{subfiles}
% このファイル内だけのコマンド
\begin{document}
\begin{center}
\Large{概要}
\end{center}
\normalsize
\thispagestyle{empty}
バクテリアや魚、鳥などのように、自己駆動する要素の集団をアクティブマターと呼ぶ。
アクティブマターは典型的な非平衡系であり、平衡系とは異なる様々な現象が見られる。
% その研究対象の1つとして、障害物との相互作用
それらの現象の1つ
の典型的な例として、自発的な渦の発生が挙げられる。このような系では速度相関が発達するが、こ
の相関長はある特徴的な長さを持っており、それは渦の大きさに関係することがわかっている。
このようなアクティブマターを円の中に閉じ込めると、様々な流れが生まれる。
特に円の半径がこの特徴的な長さと同じ場合、円の中の流れは1つとなり、
1つの渦を切り出すことができる。
この現象は現在では理論、実験ともに様々な系で観測されている。従来の理論研究においては
整列相互作用を持つモデルが主に使われていた。しかし、近年そのような整列相互作用をもたない、
Active Brownian Particles (ABP)やABPにキラリティーを加えたChiral Active Brownian Particles(CABP)
のバルク系においても渦が生まれることがわかった\cite{szamelLongrangedVelocityCorrelations2021,szamelLongrangedVelocityCorrelations2021,kurodaAnomalousFluctuationsHomogeneous2023,kurodaLongrangeTranslationalOrder2024}。
しかし、それらのモデルにおける閉じ込め系の研究はまだ少なく、閉じ込めによる効果と渦の間にある関係は明らかになっていない。


本研究では、ABP 及び CABP を円形領域に閉じ込めてシミュレーションを行い、円の中に生じる
流れや渦について解析を行なった。ABP 系においては、アクティブ度を表すパラメータ Péclet 数を変化させ、
系の流れについて調べた。
Péclet 数が小さい領域では全ての粒子がランダムな方向へ
運動する無秩序な状態だったが、Péclet 数が大きくなると粒子全ての運動方向がそろい、協同的な
運動をした。その回転方向を見ると時計回り、反時計回りの回転を不規則に繰り返しており、定常的
な流れは発生しなかった。また、CABP 系においては、粒子の回転半径 $R_\Omega$ を変化させて同様に流れや渦について調べた。
その結果、$R_\Omega$ が大きくなるに従って系は空間的に無秩序になる無秩序相、
粒子全ての運動方向が揃い、規則正しく振動する振動相、全ての粒子が1つの渦となって回り続ける定流相が現れた。
これらの相が遷移するメカニズムについて考察し、無秩序相と振動相の間の遷移については
速度相関の相関長で、振動相と定流相の間の遷移は1粒子描像でそれぞれ説明できることが明らかになった。
この研究の結果は、アクティブマターの自己駆動力による相関長と閉じ込めによる効果の関係を明らかにするものである。

\end{document}

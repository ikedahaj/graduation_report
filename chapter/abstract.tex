\documentclass[/Users/ikedahajime/GitHub/reserch/master_report/thesis]{subfiles}
% このファイル内だけのコマンド
\begin{document}
\begin{center}
\Large{概要}
\end{center}
\normalsize
\thispagestyle{empty}
バクテリアや魚、鳥などのように、自らの力で進むものをアクティブマターと呼ぶ。
アクティブマターはその性質により非平衡であり、平衡系とは異なる様々な現象が見られる。
% その研究対象の1つとして、障害物との相互作用
それらの現象のうちの1つとして、自発的に渦を生むことがあげられる。
このような系では速度相関が発達するが、その相関長はある特徴的な長さを持っており、
この長さは渦の大きさに関係していることがわかっている。
%渦は様々な大きさを持つが、このような状態においては渦の大きさを表す特徴的な長さが存在する。
このようなアクティブマターを円の中に閉じ込めると、様々な流れが生まれる。
特に円の半径がこの特徴的な長さと同じ場合、円の中に1つの渦を切り出すことができる。
この現象は現在では理論、実験ともに様々な系で観測されている。理論における研究は
整列相互作用を持つモデルが主に使われていたが、近年そのような相互作用がない、Active Brownian Particles (ABP)や
ABPにキラリティーを加えたChiral Active Brownian Particles(CABP)においても渦が生まれることがわかった\cite{szamelLongrangedVelocityCorrelations2021,szamelLongrangedVelocityCorrelations2021,kurodaAnomalousFluctuationsHomogeneous2023,kurodaLongrangeTranslationalOrder2024}。
しかし、それらのモデルにおける閉じ込め系の研究はまだ存在しない。\\
 本研究では、ABP 及び CABP を円形領域に閉じ込めてシミュレーションを行い、円の中に生じる
流れや渦について解析を行なった。ABP の系においては、アクティブさ (Péclet 数) を変化させ、
系の流れについて調べた。Péclet 数が小さい領域では全ての粒子がバラバラな方向へ運動する
無秩序な状態だったが、Péclet 数が大きくなると粒子全体が1つの方向へと回転した。
その回転方向を見ると時計回り、反時計回りの回転を不規則に繰り返しており、定常的な流れは発生しなかった。
また、CABP の系においては、粒子の回転半径 $R_\Omega$ を変化させて同様に流れや渦について調べた。
その結果、$R_\Omega$ が大きくなるに従って系は空間的に無秩序になる無秩序相、
系が規則正しく振動する振動相、1つの渦となって回り続ける定流相が現れた。
これらの相がを遷移するメカニズムについて考察し、無秩序相と振動相の間の遷移については
速度相関の相関長で、振動相と定流相の間の遷移は一粒子描像でそれぞれ説明できることがわかった。

\end{document}

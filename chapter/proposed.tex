\documentclass[/Users/ikedahajime/GitHub/reserch/master_report/thesis]{subfiles}
% このファイル内だけのコマンド
\begin{document}
\chapter{手法と数値計算設定}
この章ではシミュレーションの設定と、計測したパラメータについて記す。
\section{数値シミュレーションの設定}
% TODO:章の部分を外部参照にする;
モデルには第1、2章では Active Brownian Particles(ABP)\cite{filyAthermalPhaseSeparation2012} を用いた。 ABP モデルは以下のように表される。
% ABPの方程式
\begin{eqnarray}
    \dot{\bm{r}_i}(t) &=& \zeta \bm{F}_i(t)+v_0 \bm{e}(\theta_i (t))
\end{eqnarray}
\begin{eqnarray}
    \dot{\theta_i }(t) &=& \sqrt{2/\tau}\eta_i(t)
\end{eqnarray}
ここで、\mbox{\boldmath$r$}は粒子の位置、$v_0$ は自己推進の速度、$\mu$ は揺動度を表す。
$\bm{F}_i$は相互作用を表し、二体間相互作用ポテンシャル$U$を用いて$\bm{F}_i=\sum_{i\neq j} \nabla_i\bm{U}(r_{ij})$と書ける。$\eta_i$ はホワイトノイズで、$\langle \eta_i(t) \eta_j(t') \rangle=\delta_{ij}\delta(t-t')$の関係を満たす。
ここで、$\langle \dots \rangle$ はアンサンブル平均である。
$\mbox{\boldmath$e$}(\theta_i)=(\cos\theta_i,\sin \theta_i)$は自己推進の方向を表す単位ベクトルで、
$\tau$は緩和時間である。このモデルは、各粒子が角度$\theta_i$の方向に、持続時間$\tau$の間、速度$v_0$で進むことを表す。\\
ポテンシャルには、Weeks-Chandler-Andersen(WCA)ポテンシャル\cite{weeksRoleRepulsiveForces1971}を用いた。
\begin{equation}
    U(r_{ij})=
    \begin{cases}
        4\epsilon\left(\left(\frac{\sigma}{r_{ij}}\right)^{12}-\left(\frac{\sigma}{r_{ij}}\right)^6+\frac{1}{4}\right) & (r_{ij}<2^{\frac{1}{6}}\sigma)\\
        0 &(r_{ij}>2^{\frac{1}{6}}\sigma)
    \end{cases}
\end{equation}
ここで、$r_{ij}=\left|\bm{r}_i-\bm{r}_j\right|$は粒子間距離、$\sigma$は粒径である。\\
数値計算のため、
%TODO: 章を外部参照にする;
第3章では、ABPにキラリティーに関する項を加えた、 Chiral Active Brownian Particles(CABP)\cite{teeffelenDynamicsBrownianCircle2008}TODO::aliment forceがないものに変える を用いた。このモデルは以下のように表される。
% CABPの方程式
\begin{eqnarray}
    \dot{\bm{r}_i}(t) &=& \zeta \bm{F}_i(t)+v_0 \bm{e}(\theta_i (t))
\end{eqnarray}
\begin{eqnarray}
    \dot{\theta_i }(t) &=& \Omega+\sqrt{2/\tau}\eta_i(t)
\end{eqnarray}
$\Omega$はキラリティーの強度である。最も簡単な場合として揺らぎがゼロ、つまり緩和時間$\tau=\infty$の極限を解析した。
ポテンシャルにはHarmonicポテンシャルを用いた。
\begin{equation}
    U(r_{ij})=
    \begin{cases}
        \frac{\epsilon}{2}\left(\frac{r_{ij}}{\sigma}-1\right)^2 &(r_{ij}<\sigma)\\
        0 & (r_{ij}>\sigma)

    \end{cases}
\end{equation}

\section{パラメータ}
観測するパラメータとして、以下のパラメータを用いた。
\subsection{規格化された角運動量}
規格化された角運動量V\cite{jiangEmergenceCollectiveDynamical2017}\cite{capriniCollectiveEffectsConfined2021}は以下のように定義される。
\begin{equation}
    V=\frac{1}{V}\Sigma_i \frac{L_{zi}}{ |L_{zi}|}
\end{equation}
ここで、$L_{zi}$は$i$番目の粒子の角運動量で、$L_{zi}=x_i*v_{yi}-y_i*v_{xi}$。\\
このパラメータは粒子の運動が

\subsection{渦度}
渦度場は、速度場$\bm{v}(\bm{r})$を用いて$\omega(\bm{r})=\partial_x v_y -\partial_y v_x$と表される。
数値計算上、速度場は以下のように計算する。空間を$\delta x=\delta y=0.25 \sigma$の格子状に分割する。
このとき、格子点$r(i,j)$における速度場は%この書き方でいいのか?
\begin{equation}
    \bm{v}(r(i,j))=\sum_{\left|\bm{r}_k-r(i,j)\right|<3\sigma} f(\left|\bm{r}_k-r(i,j)\right|)\bm{v}_k
\end{equation}
のように計算する。ここで、$f(r)=(2\pi s^2)^{-1}. \exp(-r^2/2s^2)$はガウシアン関数であり、分散$s$は$3/\sqrt{2\log 10}$とした。\\
この速度場を用いて、渦度場は格子点の周りの速度場を線積分することで計算される。
\begin{equation}
    \omega (r(i,j))=\frac{v_x(r(i,j))\delta x +v_y(r(i+1,j)) \delta y -v_x(r(i+1,j+1))\delta x -v_y(r(i,j+1))}{\delta x \delta y}
\end{equation}


方位角成分の運動量は以下のように定義される。
\begin{eqnarray}
    m_n &=& \int_0^Rdr r\left|\frac{1}{2\pi}\int_0^{2\pi}d\theta v(r,\theta)\right|^2
\end{eqnarray}
離散化して、
\ifSubfilesClassLoaded{
    \printbibliography[title=参考文献]
    % \printbibliography[title=参考文献]

    }{}
\end{document}
